\documentclass{article}
\usepackage[utf8]{inputenc}
\usepackage[T1]{fontenc}
\usepackage{lmodern}
\usepackage[margin=1in]{geometry}
\usepackage{hyperref}
\usepackage{graphicx}
\usepackage{amsmath}
\usepackage{url}
\urlstyle{same}

\title{Ionfyx Infrared Pain-Relief Claims – Credibility Analysis}
\author{}
\date{July 2025}

\begin{document}

\maketitle

\begin{abstract}
This document analyzes the scientific credibility of Ionfyx's technology, a line of wearable "smart fabric" patches that claim to relieve pain by emitting far-infrared (FIR) light.
\end{abstract}

\section{Introduction}

Ionfyx is a Spanish product line of wearable "smart fabric" patches that claim to relieve pain by emitting far-infrared (FIR) light. According to the inventor José Bravo, the patches contain \textbf{bioceramic nanoparticles that convert ambient ultraviolet (UV) light into FIR}, purportedly penetrating deep into tissues to improve circulation and promote healing. The company markets these neoprene-based wraps (for back, knee, etc.) as a cutting-edge solution to muscle and joint pain, boasting benefits like reduced inflammation and accelerated tissue regeneration. Ionfyx's website even asserts that the infrared from the patches penetrates the body up to 15 cm. These are extraordinary claims that invite scrutiny. This analysis examines the scientific credibility of Ionfyx's technology, addressing each of the points of skepticism raised, and cross-checks them against physics and independent sources. Key issues include the existence of any supporting patents, the availability of UV light (especially at night), UV transmission through clothing and neoprene, the actual penetration depth of FIR vs. near-infrared (NIR) in tissue, and alternative explanations (like simple warmth or placebo) for any pain relief. Each counter-argument is backed by reliable sources, and any misconceptions are corrected for accuracy.

\subsection{Claimed Technology and Patent Status}

Ionfyx describes its product as an "intelligent fabric" that eliminates pain by converting ambient UV into therapeutic infrared. The technology is said to derive from a serendipitous discovery: Bravo spent five years researching ways to harvest environmental energy (initially to generate electricity from ambient UV for use on a boat) and found a nanoparticle that emits infrared when excited by UV. This was then applied to bodily pain relief after he experienced a lumbar injury, leading to the first Ionfyx prototype. Despite the high-tech narrative, \textbf{Ionfyx provides no patent number or publication to verify the invention.} The marketing materials refer to the nanoparticle advance as if it were proprietary, yet \textbf{no record of an Ionfyx patent is found in patent databases or official records as of 2025} (the company itself does not list any patent on its site, only stating it spent years of research). This suggests that either the technology is not actually patented or any patent exists only as a pending or obscure filing. In contrast, similar fabric technologies do exist globally – for example, the U.S. company \textbf{Coolvio} markets a "patented fabric" that converts UVA/UVB sunlight to red and near-infrared light for therapeutic use in pets. However, Ionfyx's specific implementation (UV->FIR for human pain relief) does not appear in known patent literature under Bravo's name. The lack of a verifiable patent or peer-reviewed study on Ionfyx's nanoparticles is a \textbf{red flag}: it means the core claim relies solely on the inventor's anecdotal story and marketing, without independent validation of the science.

\subsection{Ambient UV Light: Day vs. Night}

Ionfyx's mechanism depends on ambient ultraviolet light, which it claims to be "always present, even in darkness", to activate its nanoparticles. This claim is \textbf{scientifically dubious}. In reality, \textbf{UV light is essentially absent at night}. Ultraviolet radiation primarily comes from the sun, and once the sun is down, the level of environmental UV is practically zero. The physics of night-time illumination show that \textbf{at night there is virtually no UV present}, and even typical artificial lighting emits negligible UV. Moonlight is just reflected sunlight and contains only trace amounts of UV (the moon's albedo for UV is low and any residual UV is largely filtered by Earth's atmosphere). In short, \textbf{Ionfyx cannot rely on UV after sundown}, which directly contradicts the company's suggestion that the product works 24/7. Unless the user is exposed to sunlight or strong UV sources, the nanoparticles would have no UV energy to convert. This undermines the claim that the device "works at night" out of thin air. Ionfyx's implication that ambient UV is available in darkness is \textbf{misleading}. It's as illogical as saying a solar panel can generate power at midnight – without sunlight (or UV), the conversion process simply doesn't run.

Additionally, consider typical indoor use: many people will wear a brace under clothing or indoors away from direct sun. \textbf{Window glass and indoor lighting} also limit UV exposure – window glass blocks most UVB and a significant portion of UVA, and common LEDs/fluorescents emit very little UV. Thus, if someone wears Ionfyx inside or under a shirt, even during daytime, the nanoparticles might receive virtually no UV stimulation. The fundamental requirement of UV light raises serious feasibility issues about Ionfyx's real-world functionality. The only scenario where the claimed UV->IR conversion could consistently occur is if the patch is directly exposed to sunlight (e.g. outdoors with the device on top of clothing). Yet Ionfyx's own usage instructions advise wearing the patch over a \textbf{thin garment} for comfort/sweat absorption, which implies it's often under another layer (and indeed many users would cover it or wear it under other clothes). In summary, Ionfyx's core premise has a \textbf{solar dependency} that is neither acknowledged clearly to customers nor plausible for round-the-clock use – a strong indication that the marketing is overselling (if not outright defying) the physics of UV availability.

\section{UV Penetration Through Clothing and Neoprene}

Even assuming daylight conditions, another basic problem is \textbf{UV penetration}: can environmental UV actually reach the nanoparticles if they are wrapped around your body? Ionfyx products are made of materials including “licra, neopreno y nanopartículas biocerámicas” (lycra, neoprene, and bioceramic nanoparticles). \textbf{Neoprene} is a thick synthetic rubber commonly used in wetsuits - and it is well-known for blocking sunlight. In fact, neoprene gear is often touted as UV protective. For example, a 5 mm to 7 mm thick neoprene wetsuit hood provides effectively \textbf{100\% protection against UV rays}. Even thinner neoprene layers greatly attenuate UV. This is intuitive: surfers don't get sunburned on skin covered by a wetsuit. Thus, if Ionfyx's nanoparticle layer is encased within neoprene (a "sandwich" design), \textbf{UV light cannot penetrate through the neoprene to activate the particles}. The Ionfyx advertising doesn't detail the exact layer arrangement, but if the nanoparticles are embedded inside the patch's material, any exterior neoprene/lycra would shield them from UV. The company's claim that the UV-driven process works through clothing is also suspect. They have stated the emitted infrared is capable of penetrating even clothing, but they ignore that UV cannot similarly penetrate clothing to reach the patch in the first place. The simple observation that one does not get a tan under clothes highlights that everyday fabrics block the majority of UV.

To quantify this: a typical white cotton T-shirt has a UPF (Ultraviolet Protection Factor) around 7, meaning it allows only about 1/7th ($\sim$14\%) of UV to pass; if the shirt is wet, UPF drops to $\sim$3 (only 1/3 of UV passes). Heavier or darker fabrics can block >98\% of UV (UPF 50+). Clothing is considered one of the most effective forms of sun protection. Therefore, if Ionfyx is worn over a shirt (as recommended), the UV filtering by that garment alone drastically reduces what reaches the patch. Now add neoprene into the mix: \textbf{Neoprene is a UV absorber and an even better blocker than cotton}. One surf gear article notes that depending on thickness and composition, neoprene hoods "vuelven buenos repelentes contra los rayos ultravioleta" and thicker neoprene yields greater UV protection. In short, \textbf{multiple layers stand between ambient UV and Ionfyx's nanoparticles}: the outside world > your clothing > the Ionfyx outer fabric (lycra) > the neoprene layer. It is extremely unlikely that significant UV radiation can penetrate all that and still drive a useful infrared emission.

Essentially, Ionfyx's design seems self-defeating: it puts UV-sensitive particles inside an opaque wrap. For the nanoparticles to work as advertised, they'd need a direct line-of-sight to UV light. If worn underneath any clothing or if the patch's outer surface isn't specifically UV-transparent, the conversion process would be starved of input. The \textbf{laws of photonics} here are straightforward – UV photons can't magically tunnel through opaque materials. This casts serious doubt on Ionfyx's ability to function except perhaps in contrived conditions (outdoors, patch worn externally with no covering). It appears the creators either \textbf{neglected the UV shielding issue} or assume users won't realize it. Either way, this undermines the credibility of the product's mechanism.

\section{Far-Infrared vs. Near-Infrared: Penetration Depth Reality}

Ionfyx emphasizes \textbf{far-infrared (FIR)} light (also called IR-C, longwave infrared) as the therapeutic output. The claim on their site that the infrared penetrates \textbf{15 cm into the body} is extreme and not supported by optical physics or biomedical literature. In fact, the \textbf{penetration depth of electromagnetic waves in tissue decreases as wavelength increases} in the infrared range. \textbf{Near-infrared (NIR)}, with shorter wavelengths around 0.8-1.0 µm, can penetrate further into tissue, while \textbf{far-infrared}, with much longer wavelengths ($\sim$3–1000 µm), is absorbed very superficially. Sources consistently show that the \textbf{deepest light penetration in tissue occurs around 800-850 nm (IR-A range)}. At those NIR wavelengths, photons can reach into the dermis and even some millimeters of subcutaneous tissue (on the order of millimeters, not centimeters). By contrast, \textbf{far-IR (IR-C) is almost entirely absorbed in the epidermis} (the outer 0.1-0.2 mm of skin). An infrared therapy company summary explains: “Longwave infrared (IR-C) energy heats the upper layer of skin (0 – 0.5 mm)”, whereas "shortwave IR (IR-A) penetrates into the lower skin layers" where blood circulation exists. In other words, \textbf{FIR radiation does not meaningfully reach deep tissues at all} - it mainly produces a surface warming effect.

Therefore, Ionfyx's statement of 15 cm penetration is \textbf{scientifically baseless}. No form of light (outside of intense X-rays or other high-energy radiation) is going to penetrate 15 cm into the body and remain coherent or effective; certainly not thermal infrared, which is readily absorbed by water in the flesh. Even \textbf{near-infrared light, at best, penetrates on the order of a few centimeters under ideal conditions}, and that too with greatly diminished intensity. For instance, one study found that an 810 nm infrared laser could transmit less than 1\% of its energy through 2.5 cm of human tissue. Another reference notes NIR (780-950 nm) might reach $\sim$2-3 cm into muscle at very low intensity after scattering, but far-IR (e.g. 10 µm wavelength) would be absorbed almost entirely at the skin surface. The ISO standard divisions of IR define IR-A (near) = 0.78-1.4 µm, IR-B (mid) = 1.4-3 µm, IR-C (far) = 3 µm-1 mm. Ionfyx's "infrarrojo lejano" refers to IR-C, which corresponds to frequencies roughly 0.3-20 THz (very low compared to visible light). These long wavelengths trigger molecular vibrations in water and tissue almost immediately upon contact, depositing energy as heat in the outermost cells. They \textbf{do not act like penetrating rays}. Thus, the \textbf{corrected penetration figures} are roughly: Near-IR: a few millimeters to perhaps 1-2 cm for the most optimal wavelengths; Far-IR: fractions of a millimeter of direct penetration.

Given this reality, if Ionfyx patches truly emit primarily FIR, the therapeutic effect would be limited to \textbf{surface warming}. To illustrate the discrepancy: Ionfyx's advertised 15 cm depth is an overestimate by \textbf{two orders of magnitude} compared to what physics and biology show for IR-C penetration. It's possible the Ionfyx team conflated penetration of heat sensation with actual photon travel – FIR can induce \textbf{indirect warming} deeper as blood circulates heat or as tissues respond reflexively, but the photons themselves aren't going that far. Even manufacturers of infrared sauna panels acknowledge that far-infrared only causes a gentle warming and sweating response from heating the skin, not deep tissue irradiation. For truly deep phototherapy, one would use near-infrared lasers or LEDs around 800-900 nm (such as those in some medical photobiomodulation devices), not passive FIR. The user's point was spot on: Ionfyx should have claimed NIR emission if they wanted a semblance of truth about depth – but their system doesn't produce NIR (since down-converting UV typically yields longer wavelengths in the IR-C band as heat/fluorescence). This is a \textbf{fundamental flaw}: the product emits the \textbf{"wrong" kind of infrared for deep action}, and the depth it quotes is a physical impossibility for that wavelength.

To compound matters, consider the patch construction again: if the nanoparticles did somehow produce FIR, and there's a neoprene layer between the emission and your skin, that FIR would be largely absorbed by the neoprene itself. Neoprene is not only UV-opaque, it is also a good \textbf{thermal insulator} - designed to trap IR (body heat) and not let it pass. Wetsuit neoprene keeps you warm by preventing your body's infrared and heat from escaping into the water. Likewise, a neoprene backing on Ionfyx would \textbf{retain most of the FIR radiation rather than transmitting it}. In essence, the patch would heat up internally (like a warm compress) and only transfer heat to the skin by direct contact, not by far-IR photons traveling deep. This scenario is no different from a common therapeutic heat pack. The conclusion here is that Ionfyx's claimed mechanism (UV in, FIR out penetrating 15 cm) is \textbf{deeply inconsistent with established optical science}. The \textbf{actual effect} of the device, if any, likely comes from much more mundane factors, which we explore next.

\section{Alternative Explanations: Warmth and Placebo Effects}

If some users experience pain relief from Ionfyx patches, there are simpler explanations than "nanoparticle magic." First and foremost is \textbf{thermal therapy}. Ionfyx devices are essentially neoprene braces that wrap around joints or muscles. \textbf{Neoprene braces are known to relieve pain by providing support, gentle compression, and above all heat retention}. Neoprene is a closed-cell foam rubber; when worn, it \textbf{traps the body's natural heat and warms the underlying area}. This localized warmth increases blood circulation (vasodilation), which can relax tight muscles and reduce joint stiffness. Medical suppliers note that neoprene supports "help retain the natural heat... promoting blood circulation" and that the combination of heat and compression can speed recovery of injured tissues. In fact, the entire premise of infrared therapy for pain (such as infrared lamps or saunas) is to increase tissue temperature slightly, which improves blood flow and can alleviate discomfort. A neoprene wrap achieves a similar end by insulation rather than by active radiation. So it's very plausible that Ionfyx patches work no better than a standard neoprene sleeve or heating pad - they warm the area, and the resulting improved circulation provides relief in some cases (especially muscular aches or chronic joint pain that responds to heat). Notably, one of the positive Ionfyx reviews even mentions that despite the hot weather, they keep wearing it, implying the product does feel like a warm wrap. The user's argument that one is basically putting an extra layer of clothing on the hurt area is supported by how neoprene operates. So, \textbf{Occam's Razor suggests that any genuine pain relief from Ionfyx is likely due to thermal compression effects} rather than novel photonic therapy.

Secondly, the \textbf{placebo effect} cannot be overlooked, especially in pain management. Pain is a subjective experience modulated heavily by the brain. Believing that a high-tech device is treating your pain can itself cause a reduction in perceived pain. \textbf{Placebo analgesia} is well-documented: patients often experience real relief from sham treatments when they expect a positive outcome. Harvard Medical School notes that placebos “have been shown to be most effective for conditions like pain management,” and work by creating a strong mind-body expectation of improvement. In clinical trials for pain, placebo responses can be remarkably high. For example, in one migraine study a openly labeled placebo pill still achieved about \textbf{50\% of the pain reduction of the actual drug}. The ritual of using a treatment - in Ionfyx's case, strapping on a fancy “nanoparticle" patch – triggers the brain to release endorphins and other neurotransmitters that ease pain perception. Ionfyx capitalizes on a \textbf{high-tech mystique} (nanotechnology! "invisible" UV light! space-age fabric!) that can enhance expectation. Users reading 5-star testimonials and seeing a certified-looking inventor may firmly believe they have the ultimate pain cure on their body. That confidence alone can translate to feeling better. The user correctly points out that the placebo effect is very powerful - indeed, it can physiologically dampen pain signals even though no active ingredient is present. Given that Ionfyx has a strong placebo setup (novel device, personal coaching by the founder in some cases, testimonials), \textbf{some of its success stories likely owe to placebo effect}. This doesn't mean the pain relief isn't real for those people - it means the mind was tricked into self-healing to an extent, which is a credit to the power of belief, not the patch itself.

Psychological factors are also at play. Chronic pain sufferers are often anxious for hope; a new gadget with lots of praise can boost their mood and reduce anxiety, which in turn can reduce muscle tension and pain. There's also \textbf{regression to the mean} – pain flares tend to subside naturally over time, so any intervention done at a peak will look "effective" when things return to average. If someone wears Ionfyx during a recovery period, they might attribute natural healing to the device. And because Ionfyx encourages round-the-clock use, users may also be \textbf{reducing movement of the affected area} (due to wearing a brace), which helps healing; again, not due to UV-to-IR conversion but simply due to rest and stabilization.

In summary, \textbf{none of Ionfyx's positive outcomes demand exotic physics to explain}. Warm compress effect, support/compression, and placebo suggestion together provide a very plausible explanation for many of the anecdotal reports of pain relief. These are the same reasons people feel better with common braces, Ace bandages, heat packs, or even copper bracelets - not because of copper ions or nanoparticles, but because of known physiologic and psychologic effects. The danger is when a product markets these benefits under a veil of pseudoscience and charges a premium for it, which brings us to credibility and ethics.

\section{Customer Experiences and Red Flags}

Ionfyx's public reviews present a mixed picture. On one hand, the company highlights numerous glowing testimonials. For instance, on Trustpilot (where Ionfyx has an unclaimed profile), it boasts an average rating around 4.8/5 with many users reporting reduced pain and thanking José Bravo personally for his support. Some claim dramatic results (e.g. a chronic knee pain resolved in 3 months, a severe ankle sprain recovery accelerated). It's worth noting that almost all these reviews are 5-stars; this, while not impossible, is somewhat unusual and suggests Ionfyx may actively solicit positive feedback from satisfied customers. The presence of the founder's direct involvement (multiple reviewers mention “el trato de José fue muy bueno") indicates a very personal sales/support process. While happy customer testimonials are not proof of efficacy (for reasons discussed above), they show that some users are convinced the product helped them. We must interpret these with caution: \textbf{anecdotes are subject to placebo effect and confirmation bias}. Without controlled trials, we don't know how many people saw no benefit or how it compares to a dummy neoprene wrap.

On the other hand, there are \textbf{negative or cautionary experiences} that raise concerns about Ionfyx's credibility as a business. The Spanish Consumers Organization OCU has on record a complaint about Ionfyx's return/refund practices. In October 2024, a customer reported ordering a custom-sized Ionfyx lumbar brace and being told a €10 fee would be charged if returned (for the custom work). The product arrived 8 cm smaller than the requested measurements (it didn't fit at all), so the customer returned it immediately. Ionfyx still deducted €10 from the refund, even though the sizing mistake was theirs, and their website hadn't disclosed any custom fee policy. The customer protested that the fee for a garment they hadn't actually tailored correctly is unjust, and threatened to escalate to consumer authorities and public reviews if not reimbursed. Ionfyx's response was unapologetic - they insisted the modification was done "counting on the elasticity" of the material and that the customer had agreed to the charge, so they stuck by keeping the €10. This incident suggests a \textbf{lack of professionalism or fairness} in Ionfyx's customer service. Charging a custom fee for a mis-sized product and then stonewalling the customer reflects poorly. It also hints that Ionfyx might be a \textbf{small operation} (the founder possibly handling complaints himself) and not very customer-friendly when things go wrong.

Another red flag is the \textbf{price point} of Ionfyx compared to what it is. The products cost on the order of €90-€140 for a single patch or brace. Essentially, this is a piece of neoprene at the price of gold, as the user aptly put it. For comparison, a generic neoprene knee or back brace can cost anywhere from €15 to €50, and an electric IR heating pad perhaps €30-€60. Ionfyx is \textbf{significantly more expensive}, presumably due to the “nanoparticle” gimmick. Without clear evidence that it outperforms a regular heat/compression therapy, this pricing is hard to justify. Overpricing in itself doesn't mean it's a scam, but \textbf{when a product's cost is high and its scientific rationale is shaky, it leans into exploitation}. They are potentially targeting people with chronic pain (a vulnerable group desperate for relief) and upselling a glorified support brace as advanced tech.

The \textbf{credibility of the health claims} is also suspect given no clinical trials. Ionfyx cites general studies about infrared's benefits (even claiming “más de 8,000,000 de estudios" on infrared light's positive effects and linking to a broad PubMed search). While it's true infrared therapy has known benefits in certain contexts, Ionfyx extrapolates this to imply their specific product is proven. In reality, \textbf{none of those millions of studies are about Ionfyx itself}. Many are about \textbf{active phototherapy devices} (lasers, LEDs) or \textbf{sauna heat} - not passive fabrics converting UV. Ionfyx also links a recent review on ceramic infrared-emitting materials, which indeed discusses bioceramic fabrics showing some improved recovery in athletes. However, such studies often involve \textbf{specialized conditions (e.g. fabrics that reflect the body's own IR)}, and results are modest and sometimes conflicting. No peer-reviewed publication is offered that demonstrates Ionfyx patches curing injuries or reducing pain beyond placebo. The \textbf{lack of independent testing} is a serious gap for a health product making therapeutic claims. In many jurisdictions, if a product is marketed for pain relief, it might even need regulatory clearance (like FDA approval as a medical device in the US or a CE mark for medical devices in Europe). Ionfyx does not mention any such approvals. It styles itself as a sort of sports/consumer item, which might be to avoid medical regulatory scrutiny. This is another hallmark of dubious health gadgets: they occupy a grey area where they don't have to clinically prove their efficacy.

In social media, Ionfyx has tried to gain traction (the user shared a TikTok link presumably showing Ionfyx promotions). There have also been \textbf{negative discussions on platforms like TikTok labeled "Ionfyx Opiniones Negativas”}, which aim to reveal the reality behind the hype. The very fact that such skepticism is spreading indicates Ionfyx's claims are being challenged by some critical voices. It's common for scam or pseudo-science products to have a wave of initial promotional reviews followed by growing skepticism as more people question or fail to get results.

\section{Conclusion}

In conclusion, \textbf{Ionfyx's credibility is highly suspect when examined through a a scientific lens}. The product's mechanism as advertised is fraught with implausibilities: it purportedly relies on UV light that isn't available at night or under clothing, it claims to emit far-infrared that somehow penetrates inches into the body (contradicted by established biophysics, which shows FIR only heats the skin surface), and it ignores that its own construction (neoprene) would block the very energies involved. No supporting patent or clinical research is presented to substantiate the extraordinary claims, and none could be found in public records. On the other hand, the aspects of Ionfyx that do "work" are explainable by conventional means: the neoprene wrap warms and supports the area, improving comfort and blood flow, and the user's belief in the high-tech device triggers a potent placebo pain relief response. These effects are real but not unique to Ionfyx's technology - they could be achieved with much cheaper and simpler remedies.

The \textbf{balance of evidence suggests that Ionfyx is more of a cleverly marketed gadget than a revolutionary therapy}. While it may not be a scam in the sense of completely non-functional (people do feel some relief, just as they would with any warm compress), it is \textbf{scamming consumers in terms of scientific honesty and value}. It is sold at a steep price with claims that do not hold up under scrutiny, which is a hallmark of health fraud. Moreover, at least one customer complaint highlights less-than-stellar business practices regarding returns. Potential buyers should be extremely cautious and not be swayed by the hype. If one is simply looking for pain relief, \textbf{evidence-based options} like medically approved red/NIR light therapy devices, physical therapy exercises, or even standard heat pads have more transparent credentials. At minimum, a basic neoprene brace (at a fraction of the cost) will provide the same supportive warmth that likely underpins Ionfyx's benefits.

In summary, \textbf{Ionfyx's narrative of nanoparticle UV-to-IR healing is not supported by physics or credible evidence}. It's an example of how scientific terms can be strung together to impress consumers without delivering on those principles in practice. The product is real in a physical sense (it's a wrap you can wear), but the \textbf{promised mechanism is borderline fantasy}. Thus, it is fair to say Ionfyx falls into the category of a modern \textbf{health scam} – leveraging the placebo effect and basic heat therapy, but cloaked in pseudoscience to justify an exorbitant price. Consumers and patients should demand robust proof for such claims, and in this case, the proof is sorely lacking. As the saying goes, if it sounds too good to be true, it probably isn't true.

\section*{Sources}
The analysis above is supported by scientific and industry sources that cover UV and IR physics, material properties, and medical insights: ambient UV absence at night, UV blockage by fabrics and neoprene, infrared penetration depths in tissue, the physiological effects of neoprene braces (heat and circulation), and the documented power of placebo in pain relief. Additionally, a consumer complaint via OCU illustrates Ionfyx's questionable refund practice. All these pieces paint a consistent picture that starkly contradicts Ionfyx's marketing claims.

\begin{itemize}
    \item Sobre Ionfyx - Ionfyx: \url{https://ionfyx.com/sobre-ionfyx}
    \item Investigadores españoles crean un tejido inteligente que estimula al organismo para que se recupere más rápido: \url{https://www.infosalus.com/salud-investigacion/noticia-investigadores-espanoles-crean-tejido-inteligente-estimula-organismo-recupere-mas-rapido-20190219140141.html}
    \item The Science Behind Coolvio® Patented Fabric Technology: \url{https://coolvio.com/pages/light_therapy_for_dogs}
    \item Ultraviolet light in dark environments - Physics Stack Exchange: \url{https://physics.stackexchange.com/questions/214796/ultraviolet-light-in-dark-environments}
    \item rodillera-ion - Ionfyx: \url{https://ionfyx.com/tienda/rodillera-ion}
    \item ¿Usar gorritos de neopreno realmente nos protege de los rayos UV? | Olas Perú, Reporte de mar, Noticias de Surf: \url{https://www.olasperu.com/noticias/surf/06082016-1/usar-gorritos-de-neopreno-realmente-nos-protege-de-los-rayos-uv}
    \item Sun Protective Clothing: \url{https://www.skincancer.org/skin-cancer-prevention/sun-protection/sun-protective-clothing/}
    \item Physical properties and biological effects of ceramic materials emitting infrared radiation for pain, muscular activity, and musculoskeletal conditions - PMC: \url{https://pmc.ncbi.nlm.nih.gov/articles/PMC10084378/}
    \item Types of infrared heaters in an infrared cabin | Infraworld: \url{https://www.infraworld.com/en/useful-information/types-of-infrared-heaters}
    \item Can infrared light really be doing what we claim it is doing ... - Frontiers: \url{https://www.frontiersin.org/journals/neurology/articles/10.3389/fneur.2024.1398894/full}
    \item Neoprene Knee Sleeve | Neoprene Knee Braces, Neoprene Knee Support: \url{https://www.braceability.com/collections/neoprene-knee-braces}
    \item Opiniones sobre Ionfyx | Lee las opiniones sobre el servicio de ionfyx.com: \url{https://es.trustpilot.com/review/ionfyx.com}
    \item The power of the placebo effect - Harvard Health: \url{https://www.health.harvard.edu/newsletter_article/the-power-of-the-placebo-effect}
    \item PROBLEMA CON EL REMBOLSO - Reclamaciones y opiniones IONFYX - 15/10/2024: \url{https://www.ocu.org/reclamar/lista-reclamaciones-publicas/problema-con-el-rembolso/556d6f64a6f5fe28dd}
    \item Ionfyx - Elimina el dolor con infrarrojos: \url{https://ionfyx.com/}
    \item Ionfyx Opiniones Negativas - TikTok: \url{https://www.tiktok.com/discover/ionfyx-opiniones-negativas}
\end{itemize}

\end{document}
